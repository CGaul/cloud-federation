\chapter{Network-Federation}
- This is the first chapter, referencing the implementation part of the MA.
- The implementation part is divided into Network-Federation (federation on a basic level, using OpenFlow) and Cloud-Federation (in the upcoming chapter).


\section{OpenFlow}
- What is OpenFlow?
- Why is it so useful for a Federated environment?
- Architectural Description: OF-Controller / OF-Switches
- Some flow descriptions (handover to controller, etc.)
- Basics on OpenFlow Table entries
- OpenFlow enabled Switches -> bridge passage to OpenVSwitch



\section{OpenVSwitch}
- OpenVSwitch as a OpenFlow enabled, virtual Switch
- OpenVSwitch functionality with / without OFC

\subsection{GRE-Tunnels}
- GRE-Tunnels in general (what is Generic Routing Encapsulation? Small header description)
- GRE-Tunnels in conjunction with OpenVSwitches -> Build up Gateway Switches



\section{Mininet}
- Mininet as a Network Simulator, using OpenVSwitches and OFC(s)
- How could Mininets be federated?
- Pros/Cons \& Limitations of using Mininet as a replacement of a real network simulation, using physical devices.